\newcommand*{\ie}{\textit{i.e.\ }}
\newcommand*{\etc}{\textit{etc.}}
\newcommand*{\eg}{\textit{e.g.\ }}
\newcommand*{\etal}{\textit{et al.\ }}

\newcommand*{\abs}[1]{\left\lvert #1\right\rvert}
\newcommand*{\bra}[1]{\left\langle #1 \right\rvert}
\newcommand*{\ket}[1]{\left\lvert #1 \right\rangle}
\newcommand*{\bket}[2]{\left\langle \, #1 \,|\, #2 \, \right\rangle}
\newcommand*{\boket}[3]{\langle\, #1 \,|\, #2 \,|\, #3 \,\rangle}
\newcommand*{\com}[2]{\left[#1,#2\right]}

\newcommand*{\ten}[1]{\!\times\!10^{#1}}


\newcommand{\onlinecite}{\citenum}
\newcommand{\lin}[1]{{\skolarlining #1}}
\newcommand{\thought}[1]{{\skolarSC #1}}

%Label markers so there are no need for legends in a plot
\newcommand*{\plotmarker}[2]{
    \begin{tikzpicture}[baseline=-0.7ex]
        \node (0,0) [scale=#1, #2] {};
        \end{tikzpicture}
}
%Same deal but for lines
\newcommand*{\plotline}[1]{
    \begin{tikzpicture}[baseline=-0.7ex]
        \draw [#1] (0,0) -- (0.5,0);
        \end{tikzpicture}
}