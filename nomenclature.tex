\chapter{Nomenclature}
\renewcommand*{\nompreamble}{%
    \begin{list}{}{%
    \renewcommand{\makelabel}[1]{\eqparbox[b]{listlab}{##1}}%
    \setlength{\labelwidth}{\eqboxwidth{listlab}}%
    \setlength{\labelsep}{4pt}%
    \setlength{\parsep}{3pt}%
%    \setlength{\itemsep}{0pt}%
    \setlength{\leftmargin}{\labelwidth+\labelsep}%
    \setlength{\rightmargin}{0pt}}%
    \rightskip\RaggedRightRightskip%
}
\renewcommand*{\nomornmnt}{\resizebox{!}{1cm}{\includestandalone{figures/flourish}}} %Separator
\renewcommand*{\nompostamble}{\end{list}}
\renewcommand*{\nomenlist}[1]{\cref{#1}}
\renewcommand*{\nomgroupfont}[1]{{\large\textit{#1:}}}
\renewcommand*{\nomformatlbl}[1]{\color{lightgray}\fbox{#1}}
\begin{SingleSpace}
%Rule of thumb, if there are three or fewer sections within a chapter, reference them all, if four or more - reference the chapter
% run
% makeindex -s nomenlev.ist -o thesis.nls thesis.nlo
% to sort the nomenclature entries into alphabetic order
%    \nomdef{key}{symbol}{explanation}, \nomref{key}{label}, \nomdref{key}{symbol}{explanation}{label} to introduce symbols. (nomdref is a shortcut for doing both nomdef and nomref). The "key" field should start with:
%        A:Abbreviations
%        B:Symbols
%        C:Latin Letters
%        G:Greek Letters
%        X:Superscripts
%        Z:Subscripts
%    and will be used for sorting by makeindex (so be especially careful with Greek letters to get the proper alphabetization). In draft mode, the "key" field is shown in the margin.
%    To alphabetize the Greek properly, I use the conventions:
%        a=alpha
%        b=beta
%        c=gamma
%        d=delta
%        e=epsilon
%        f=zeta
%        k=kappa
%        o=xi
%        r=rho
%        s=sigma
%        w=phi
%        x=chi
%        z=omega
\printnomens
\end{SingleSpace} 