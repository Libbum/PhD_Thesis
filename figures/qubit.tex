\documentclass[tikz]{standalone}
\usepackage[american currents,cuteinductors]{circuitikz}
\usepackage{mathspec} %loads fontspec
\setromanfont{Skolar PE}[Ligatures={TeX, Common}, Numbers=OldStyle]
%Override shitty JJ and Squid designs
\makeatletter
%% SQUID added by Tim DuBois 10 Oct 2011
\pgfcircdeclarebipole{}{\ctikzvalof{bipoles/squid/height}}{squid}{\ctikzvalof{bipoles/squid/height}}{\ctikzvalof{bipoles/squid/width}}{

	\pgfsetlinewidth{\pgfkeysvalueof{/tikz/circuitikz/bipoles/thickness}\pgfstartlinewidth}
	
    \pgfpathmoveto{\pgfpoint{0.35*\pgf@circ@res@left+0.5*\pgf@circ@res@left}{0.35*\pgf@circ@res@up}}
	\pgfpathlineto{\pgfpoint{0.35*\pgf@circ@res@right+0.5*\pgf@circ@res@left}{0.35*\pgf@circ@res@down}}
	\pgfpathmoveto{\pgfpoint{0.35*\pgf@circ@res@left+0.5*\pgf@circ@res@left}{0.35*\pgf@circ@res@down}}
	\pgfpathlineto{\pgfpoint{0.35*\pgf@circ@res@right+0.5*\pgf@circ@res@left}{0.35*\pgf@circ@res@up}}

    \pgfpathmoveto{\pgfpoint{0.35*\pgf@circ@res@left+0.5*\pgf@circ@res@right}{0.35*\pgf@circ@res@up}}
	\pgfpathlineto{\pgfpoint{0.35*\pgf@circ@res@right+0.5*\pgf@circ@res@right}{0.35*\pgf@circ@res@down}}
	\pgfpathmoveto{\pgfpoint{0.35*\pgf@circ@res@left+0.5*\pgf@circ@res@right}{0.35*\pgf@circ@res@down}}
	\pgfpathlineto{\pgfpoint{0.35*\pgf@circ@res@right+0.5*\pgf@circ@res@right}{0.35*\pgf@circ@res@up}}

	\pgfusepath{draw}

    \pgfsetlinewidth{\pgfkeysvalueof{/tikz/circuitikz/bipoles/thickness}*0.5\pgfstartlinewidth}
    \pgfpathmoveto{\pgfpoint{\pgf@circ@res@left}{\pgf@circ@res@step}}
	\pgfpathlineto{\pgfpoint{\pgf@circ@res@right}{\pgf@circ@res@step}}
    \pgfusepath{draw}
}
%% Generic barrier added by Tim DuBois 10 Oct 2011
\pgfcircdeclarebipole{}{\ctikzvalof{bipoles/barrier/height}}{barrier}{\ctikzvalof{bipoles/barrier/height}}{\ctikzvalof{bipoles/barrier/width}}{

	\pgfsetlinewidth{\pgfkeysvalueof{/tikz/circuitikz/bipoles/thickness}\pgfstartlinewidth}
    \pgfpathmoveto{\pgfpoint{0.35*\pgf@circ@res@left}{0.35*\pgf@circ@res@up}}
	\pgfpathlineto{\pgfpoint{0.35*\pgf@circ@res@right}{0.35*\pgf@circ@res@down}}
	\pgfpathmoveto{\pgfpoint{0.35*\pgf@circ@res@left}{0.35*\pgf@circ@res@down}}
	\pgfpathlineto{\pgfpoint{0.35*\pgf@circ@res@right}{0.35*\pgf@circ@res@up}}

	\pgfusepath{draw}

    \pgfsetlinewidth{\pgfkeysvalueof{/tikz/circuitikz/bipoles/thickness}*0.5\pgfstartlinewidth}
    \pgfpathmoveto{\pgfpoint{\pgf@circ@res@left}{\pgf@circ@res@step}}
	\pgfpathlineto{\pgfpoint{\pgf@circ@res@right}{\pgf@circ@res@step}}
    \pgfusepath{draw}
}
\makeatother

%Lisenfeld PRB 81 100511 (2010) diagram in tikz.
\begin{document}
\begin{circuitikz}
    \ctikzset{bipoles/cuteinductor/coils=10,
              bipoles/cuteinductor/width=1.2,
              bipoles/barrier/height=1,
              bipoles/barrier/width=1}
    \node at (-0.75,1) [rotate=90] {microwave drive};
    \node at (3,2.35) {qubit};
    \node at (6.25,2.35) {DC SQUID};
    \node at (5.25,-0.55) {flux bias};
    \draw
    (0,0) node[ground] {}
        to [sI]   (0,2) 
        to [C,-*] (2,2)
        to [C]    (2,0) -- (3,0)
        to [barrier,*-*] (3,2) -- (4,2)
        to [L]    (4,0) -- (3,0)
    (3,0) node[ground] {} to (3,0)
    (2,2) -- (3,2);
    \ctikzset{bipoles/isourceam/height=0.3,
              bipoles/isourceam/width=0.3,
              bipoles/cuteinductor/coils=5,
              bipoles/cuteinductor/width=0.4,
              monopoles/ground/width=0.18}
    \draw
    (4.8,0.1) node[ground] {}
        to [L]    (4.8,0.8) -- (5.8,0.8)
    (5.8,0.1) node[ground] {}
        to [I]    (5.8,0.8);
    \ctikzset{bipoles/cuteinductor/coils=7,
              bipoles/cuteinductor/width=0.6,
              bipoles/barrier/height=0.6,
              bipoles/barrier/width=0.6,
              tripoles/plain amp/width=0.6,
              tripoles/plain amp/height=0.4}
    \draw
    (4.8,1) [L] to (4.8,2) -- (5.8,2)
        to [squid] (6.8,2) -- (6.8,1)
        to [barrier] (5.8,1) -- (4.8,1)
    (5.4,1.5) node[ground] {} -- (5,1.5)
    (6.8,1.5) node[circ] {} -- (8,1.5)
    (7.4,0.5) node[ground] {}
        to [I, -*]     (7.4,1.5)
    (8.5,1) node[plain amp] (amp) {}
    (8.08,0.5) node[ground] {} -| (amp.+)
    (8,1.5) -| (amp.-);
\end{circuitikz}
\end{document} 