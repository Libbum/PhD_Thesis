\documentclass[tikz]{standalone}
\usepackage{fontspec}
\setromanfont{Skolar PE}[Ligatures={TeX, Common}, Numbers=OldStyle]
\newfontfamily\chinese[Scale=1.5]{Adobe Kaiti Std R}
\begin{document}
\begin{tikzpicture}
\node at (-0.8,-0.6) {\fontsize{90}{90}\selectfont\color{gray!30}{“}};
\node at (0.9,-5.3) {\fontsize{90}{90}\selectfont\color{gray!30}{”}};
%\node at (1,0) {\chinese 孔夫子};

\node at (0,0) {\chinese 人觀子};
\node at (0,-0.6) {\chinese 焉其曰};\node at (0.8,-0.78) {\tiny\chinese ,};
\node at (0,-1.2) {\chinese 廋所視};
\node at (0,-1.8) {\chinese 哉由其};\node at (-0.15,-1.98) {\tiny\chinese ,};\node at (0.3,-1.98) {\tiny\chinese ,};
\node at (0,-2.4) {\chinese 人察所};
\node at (0,-3.0) {\chinese 焉其以};\node at (0.88,-3.18) {\tiny\chinese ,};
\node at (-0.25,-3.6) {\chinese 廋所};
\node at (-0.25,-4.2) {\chinese 哉安};\node at (-0.15,-4.38) {\tiny\chinese 。};\node at (0.3,-4.38) {\tiny\chinese 。};

\node[rotate=90] at (-1.2,-4.5) {---};
\node at (-1.2,-5) {\chinese 孔};
\node at (-1.2,-5.6) {\chinese 夫};
\node at (-1.2,-6.2) {\chinese 子};
\end{tikzpicture}
\end{document} 
