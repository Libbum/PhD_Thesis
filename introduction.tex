\chapter{Introduction}

\section{Defects in Corundum}
Various defects in corundum have been thoroughly investigated, both experimentally and theoretically for quite some time.
The first \emph{ab initio} corundum papers discussed the partially covalent nature of the aluminium -- oxygen bond and the inherent computational complexity of studying this structure~\cite{Causa1987}.
Experimental identification of defects like the $F-$ and $F^+-$ electron centres~\cite{Kotomin1989} and observations of defect mobility~\cite{Kulis1991} motivated further theoretical study.

Intrinsic defects (such as self-trapped and defect-trapped holes, or oxygen and $V^{2-}$, $V^-$ aluminium vacancies) as well as impurities of transition-metal ions like Co, Fe, Mg, Mn or Ti are all possible~\cite{Jacobs1994}, although impurities are not considered further here due to our sole interest in contaminant free structures.
\citeauthor{Jacobs1994} also suggests that oxygen vacancies are mobile throughout the lattice, as effective charges of an oxygen at a saddle point between aluminium atoms is similar to one on a normal lattice site.
However, this result is speculative for us, as simple Buckingham pair potentials are used to obtain this result (the rest of the defect calculations in the paper use a self-consistent Hartree-Fock LCAO pseudopotential).

DFT studies of single oxygen vacancies soon followed, using 120 atom supercells of corundum~\cite{Xu1997} which show the oxygen vacancy introduces a deep and doubly occupied (electronic) defect level (an $F$ center), and was found to be ``not so localized''.
Singly occupied defect levels (the $F^+$ center) are also discussed by introduction a positive background charge to the calculation.

Still to discuss: \cite{Xu1997,Heuer1999,Matsunaga2003,Carrasco2004,Perevalov2010,Pustovarov2011,Elliot2013}

\section{Defects in Amorphous Aluminium}

(testing citeauthor) \citeauthor{Momida2011} said some good stuff as did \citeauthor{Blochl1994}.
