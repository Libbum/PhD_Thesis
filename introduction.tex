\chapter{Introduction}

\section{Defects in Corundum}
Various defects in corundum have been thoroughly investigated, both experimentally and theoretically for quite some time. The first \emph{ab initio} corundum papers discussed the partially covalent nature of the aluminium -- oxygen bond and the inherent computational complexity of studying this structure~\cite{Causa1987}. Experimental identification of defects like the $F-$ and $F^+-$ centres~\cite{Kotomin1989} and observations of defect mobility~\cite{Kulis1991} motivated further theoretical study. 

Intrinsic defects (such as self-trapped and defect-trapped holes, or oxygen and aluminium vacancies) as well as impurities of transition-metal ions like Co, Fe, Mg, Mn or Ti 
are all possible~\cite{Jacobs1994}, although impurities are not considered further here due to our sole interest in contaminant free structures.

\cite{Xu1997,Heuer1999,Matsunaga2003,Carrasco2004,Perevalov2010,Pustovarov2011,Elliot2013}

\section{Defects in Amorphous Aluminium}

\citeauthor{Momida2011} said some good stuff \cite{Momida2011,Mathews2004} as did \citeauthor{Blochl1994}.  