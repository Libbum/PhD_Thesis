\versoquote{Enfermé dans le navire, d'où on n'échappe pas, le fou est confiné à la rivière aux mille bras, à la mer aux mille chemins, à cette grande incertitude extérieure à tout. Il est prisonnier au milieu de la plus libre, de la plus ouverte des routes: solidement enchaîné à l'infini carrefour. Il est le Passager par excellence, c'est-à-dire le prisonnier du passage. Et la terre sur laquelle il abordera, on ne la connaît pas, tout comme on ne sait pas, quand il prend pied, de quelle terre il vient. Il n'a sa vérité et sa patrie que dans cette étendue inféconde entre deux terres qui ne peuvent lui appartenir.}{Michel Foucault}
%Confined on the ship, from which there is no escape, the madman is delivered to the river with its thousand arms, the sea with its thousand roads, to that great uncertainty external to everything. He is a prisoner in the midst of what is the freest, the openest of routes: bound fast at the infinite crossroads. He is the Passenger par excellence: that is, the prisoner of the passage. And the land he will come to is unknown—as is, once he disembarks, the land from which he comes. He has his truth and his homeland only in that fruitless expanse between two countries that cannot belong to him. -- Michel Foucault, Madness and Civilization: A History of Insanity in the Age of Reason
\chapter{Conclusions}\label{ch:conclusions}
\chapterprecis{Thesis summary, relevant conclusions and general outlook.}

\thought{The current state of strongly} coupled TLS models represents an embarrassment of riches: copious seemingly valid ideas in both phenomenological and microscopic flavours.
It's possible that the phenomenological indistinguishability problem discussed in \cref{sec:phenom} may be circumvented with the recent emergence of the microscopic descriptions.
Furthermore, the ultimate reality of the situation may be that many of the microscopic models are not competing, but each represent a certain percentage of the total noise that presently mires Josephson junction devices.
However, several of these models fail to account for the drastic reduction in TLS counts in epitaxial junctions~\cite{Oh2006} examined in \cref{sec:introsummary}.
This thesis has attempted to identify how an 80\% reduction in visible TLSs can be observed solely by altering the JJ tunnel barrier architecture from amorphous to crystalline in nature.

Initially, we constructed atomic Josepson junction models using a hybrid \textit{ab initio} and molecular mechanics approach in \cref{ch:junctions} with various stoichiometry and density properties to reflect experimental observations of the barrier's makeup.
The structure of the amorphous layer has been historically difficult to capture computationally, although our analysis shows excellent agreement with experimentally obtained coordination values.
These junctions not only assist the analysis in this thesis, but also offer a tool for other microscopic models.
For example, hydrogen could be introduced into the structure to investigate the dangling bond hypothesis.

Starting from the charge dipole phenomenological model, \cref{ch:tise} constructs the framework required to describe an oxygen atom which has the capacity to become spatially delocalised as bonds perturb away from a crystalline structure and become amorphous.
Care was taken in choosing an empirical potential capable of describing both metallic and insulator regions due to the varying charge states an oxygen atom observes when present in a predominantly metallic environment (such as the metal--oxide boundary of a JJ).
Treatment of errors introduced via the central difference method were also studied in detail, as the models precision (resolving splitting energies around $10$ kHz) is of paramount importance.
The resolution of \textit{ab initio} methods such as DFT for comparison, has a lower bound of order $120$ GHz---well above the scale of crystal defect energies and therefore too imprecise to be considered useful for JJ study other than providing atomic positions as input.

\Cref{ch:tlslow} probes the capabilities of the model in low dimensions (\lin{1D} and \lin{2D}), subsequently \cref{ch:tlsphase} expands the dimensionality of the model to \lin{2+1D} and compares its results to current experimental strongly coupled TLS measurements in phase qubits.
The model shows its capacity to explain how an oxygen atom can generate a large dipole and appropriate ground to first excited state splitting values expected of a TLS by merely migrating from its preferred lattice position.
A prediction concerning a quad-degenerate rather than a doubly-degenerate ground state in very rare circumstances permits a possible validation method for experiments to scrutinise.
Possible explanations of the relatively small range of observed qubit-TLS coupling strengths and behaviour when in contact with an external strain field are also presented.

To examine a complete three dimensional representation of the TLS model, \cref{ch:threedee} outlines the pitfalls that memory limitations present us when using direct diagonalisaton approach in \lin{3D}.
It introduces an alternate method exploiting a Wick-rotated time-dependent Schrödinger equation to obtain time-independent ground state wavefunctions, using Gram-Schmidt orthogonalisation to extract excited states.
Equivalences between the low dimensional and \lin{3D} approaches were examined and found that the \lin{3D} solution is optimal when accurate values of TLS properties are required and the \lin{2+1D} solution to be more efficient when trends over large parameter ranges are needed.
A Voronoi classification scheme is employed, which mitigates free parameter concerns and can be used to successfully describe the potential landscape in the vicinity of a defect region or oxygen atom residing in a JJ model.
Properties of such atoms are investigated, as is the influence of external strain on a crystal of \ce{Al_2O_3}.

\divtext

There are multiple ways in which the work contained in this thesis can be expanded upon to gain a better understanding of the strongly coupled TLS and perhaps verify the possible connection to the $1/f\,$ noise phenomenon.

Active research into a procedure which mimics the formation process of the Josephson junction tunnel barrier is already underway by M.~Cyster as stated in \cref{ch:junctions}.
It is unclear if the melt and quench cycle discussed in this thesis ignores any relevant physics that may hold the key to environmental noise in these systems.
Nucleation sites near the metal surface may be an example of such a phenomenon.

A complete investigation into TLS-phonon coupling may shed more light on the strain experiments~\cite{Grabovskij2012} and extend the calculable properties of the delocalised oxygen TLS model to include dephasing and decoherence times.
Similarly, a TLS-TLS interaction model such as the one proposed by \citeauthor{Faoro2014}~\cite{Faoro2014} could be used to investigate the coherent interactions observed in \onlinecite{Lisenfeld2015}.
Further study focusing on the mapping between model dimensions would also be a fruitful endeavour.
Whilst the STM continues to produce relevant information for TLS research in a concise and analytic manner, realistic values for the tunneling energy $\Delta$ and asymmetry energy $\epsilon$ could be obtained from the delocalised oxygen model rather than assuming some arbitrary amount if a consistent transformation of variables from the \lin{3D} implementation could be projected down to \lin{1D}.

Experimentally, it is suggested that the community moves away from ever cleaner fabrication processes and focuses its attempts on implementing more accessible methods of obtaining epitaxial junctions similar to \citeauthor{Oh2006}~\cite{Oh2006}.
An alternate route may be to investigate the substitution of oxygen to a heavier non metal in the insulating layer: lowering the tunneling probability of the atom regardless of the amorphous structure it resides in.

Or, perhaps a more pragmatic approach is to continue the active research into transmons with their small surface area and minimal TLS interactions.
Perhaps also, the quantum phase slip junction may completely eradicate the need for this investigation all together.
Then it seems that the road to take boils down to a personal philosophy: do you prefer to run from the unknown or tackle it head on?
\begin{center}
\resizebox{!}{1cm}{\includestandalone{figures/sf}}
\end{center}
