\versoimage
\chapter{Exploring TLS Parametre Space}\label{ch:implementation}
\loftchap{Exploring TLS Parametre Space}
\chapterprecis{Identifying possible TLS candidates by comparing model results to experimental coupling variables such as the ground to first excited state energy splitting and dipole moment.}

\begin{figure}[htp]
\centering
\resizebox{0.8\columnwidth}{!}{\includestandalone{figures/mexhatproj}}
\caption[Potential Projections]{\label{fig:mexhatproj}\lin{1D} double wells (blue, solid) and harmonic wells (red, dashed) can be used to represent simple projections of a \lin{2D} potential onto the $x$ and $y$ axes. Left: two projected double wells is an example of a tetra-well. Right: a combination of one double well and a harmonic well reflects the hemi-tetra- case.}
\end{figure}

Possible sections:

\section{2D model}\label{sec:2d}
\subsection{Defects as Perturbed Bond Angles in a Lattice (1D)}
\subsection{TLS Defect Confined in 2 Dimensions}
\subsection{TLS Defect Confined in 3 Dimensions}
\subsection{Charge Dipoles}
\subsection{Qubit Coupling}
\section{3D}
\subsection{Methods of implementation}
\subsection{Potential Landscape analysis}
\subsubsection{Cluster analysis on a known grid}
\subsection{Comparative results with 2D model}

\section{Notes for 3D portion}
The methods outlined in \cite{Strickland2010} give us what we want for the most part, but needs to be extended a fair deal to be completely useful.
\begin{equation}
E[\Psi] = \frac{\sum_{x,y,z}\Psi(x,y,z,\tau)\left[\frac{1}{2}\sum_{i=1}^3\left(\mathbf{D\cdot\widehat{\Psi}}\right)-V(x,y,z)\Psi(x,y,z,\tau)\right]}{\sum_{x,y,z}\Psi(x,y,z,\tau)^2}
\end{equation}
where $\Psi$ is either the ground state ($\Psi_0$) or mapped via
\begin{equation}
 \ket{\Psi_1} \simeq \ket{\Psi_{snap}} - \ket{\Psi_0}\bket{\Psi_0}{\Psi_{snap}}
\end{equation}

\subsection{Gram-Schmidt Procedure}
* \cite{Griffiths2005} Grifiths, page 101-102 and 440 have some stuff on this but not much.
 