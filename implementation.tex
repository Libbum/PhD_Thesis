\versoimage
\chapter{Exploring TLS Parameter Space}\label{ch:implementation}
\loftchap{Exploring TLS Parameter Space}
\chapterprecis{Identifying possible TLS candidates by comparing model results to experimental coupling variables such as the ground to first excited state energy splitting and dipole moment.}

\begin{figure}[htp]
\centering
\resizebox{0.8\columnwidth}{!}{\includestandalone{figures/mexhatproj}}
\caption[Potential Projections]{\label{fig:mexhatproj}\lin{1D} double wells (blue, solid) and harmonic wells (red, dashed) can be used to represent simple projections of a \lin{2D} potential onto the $x$ and $y$ axes. Left: two projected double wells is an example of a tetra-well. Right: a combination of one double well and a harmonic well reflects the hemi-tetra- case.}
\end{figure}

Possible sections:

\section{2D model}\label{sec:2d}
\subsection{Defects as Perturbed Bond Angles in a Lattice (1D)}
\subsection{TLS Defect Confined in 2 Dimensions}
\subsection{TLS Defect Confined in 3 Dimensions}
\subsection{Charge Dipoles}
\subsection{Qubit Coupling}
\section{3D}
\subsection{Methods of implementation}
\subsection{Potential Landscape analysis}
\subsubsection{Cluster analysis on a known grid}
\subsection{Comparative results with 2D model}

\section{Notes for 3D portion}
The methods outlined in \cite{Strickland2010} give us what we want for the most part, but needs to be extended a fair deal to be completely useful.
\begin{equation}
E[\Psi] = \frac{\sum_{x,y,z}\Psi(x,y,z,\tau)\left[\frac{1}{2}\sum_{i=1}^3\left(\mathbf{D\cdot\widehat{\Psi}}\right)-V(x,y,z)\Psi(x,y,z,\tau)\right]}{\sum_{x,y,z}\Psi(x,y,z,\tau)^2}
\end{equation}
where $\Psi$ is either the ground state ($\Psi_0$) or mapped via
\begin{equation}
 \ket{\Psi_1} \simeq \ket{\Psi_{snap}} - \ket{\Psi_0}\bket{\Psi_0}{\Psi_{snap}}
\end{equation}

\subsection{Gram-Schmidt Procedure}

One important property of a hermitian operator (in our case $H$), is orthogonality. That is: a system's eigenfunctions belong to distinct eigenvalues which are orthogonal.
The proof of this theorem neglects degenerate states, although it can be shown that any linear combination of eigenfunctions which share the same eigenvalue are indeed eigenfunctions of the system.

Using the Gram-Schmidt orthogonalisation procedure \cite{Gram1883, Schmidt1907}, orthogonal eigenfunctions within a degenerate subspace can be constructed; \emph{chosen} to be orthogonal to the system's basis.

Consider a non-orthonormal basis of vectors $\ket{a_1}, \ket{a_1}, \ldots \ket{a_n}$.
A new, orthonormal basis for these vectors can by found be first normalising the initial vector
\begin{equation}
\ket{a^\prime_1} = \frac{\ket{a_1}}{\norm{a_1}},
\end{equation}

then finding the projection of the second vector along the first and subtracting it.
This vector is orthogonal to $\ket{a^\prime_1}$ and normalising it obtains
\begin{equation}
\ket{a^\prime_2} = \frac{\ket{a_2}-\ket{a^\prime_1}\bket{a^\prime_1}{a_2}}{\norm[\big]{\ket{a_2}-\ket{a^\prime_1}\bket{a^\prime_1}{a_2}}}.
\end{equation}

Similarly, the third vector requires the same treatment, although projections along both $\ket{a^\prime_1}$ and $\ket{a^\prime_2}$ are needed
\begin{equation}
\ket{a^\prime_3} = \frac{\ket{a_3}-\ket{a^\prime_1}\bket{a^\prime_1}{a_3}-\ket{a^\prime_2}\bket{a^\prime_2}{a_3}}{\norm[\big]{\ket{a_3}-\ket{a^\prime_1}\bket{a^\prime_1}{a_3}-\ket{a^\prime_2}\bket{a^\prime_2}{a_3}}}.\label{eq:gs3ex}
\end{equation}

This process is repeated until the orthonormal vector $\ket{a^\prime_n}$ is calculated.

Computing the excited state energies and wavefunctions using the method outlined in \xxx{section on qfdtd} relies on the orthogonality theorem, which cannot identify values in a degenerate subspace.
On the other hand, direct diagonalisation achieves the correct eigenspectrum but lacks the resolution to gain quantitative results (see \xxx{section on \lin{3D} implementation and memory issues}).
A hybrid method is therefore proposed to calculate higher order states.

Starting with a \sw{Matlab} calculated low resolution ground state wavefunction $\Psi_0^{M}$, the true excited state wavefunction $\Psi_0$ (and its' associated eigenenergy) can be calculated via \sw{qfdtd} by means of using the low resolution result as a starting guess $\Psi_0^{M} \simeq \Psi_0$.
Once $\Psi_0$ is converged, the Gram-Schmidt procedure can then be invoked to calculate the first excited state, using an initial guess from \sw{Matlab} projected along the converged, orthonormal ground state
\begin{equation}
    \ket{\Psi^\prime_1} \simeq \frac{\ket{\Psi^M_1} - \ket{\Psi^\prime_0}\bket{\Psi^\prime_0}{\Psi^M_1}}{\norm[\big]{\ket{\Psi^M_1} - \ket{\Psi^\prime_0}\bket{\Psi^\prime_0}{\Psi^M_1}}}
\end{equation}

which can again be used as input to \sw{qfdtd}.
As $\Psi^\prime_1$ has been chosen to be orthogonal to $\Psi^\prime_0$, regardless of whether the excited state exists in a degenerate subspace or not, the conversion algorithm is now able to converge to the correct eigenenergy and return an optimal excited state wavefunction.
\Cref{eq:gs3ex} can now calculate the second excited state, projecting along the converged ground and first excited wavefunctions, after which repeating the process  on higher states is possible.
% \cite{Griffiths2005} Grifiths, page 101-102 and 440 have some stuff on this but not much.

 