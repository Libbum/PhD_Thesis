\versoimage
\chapter{Exploring TLS Parametre Space}\label{ch:implementation}
\loftchap{Exploring TLS Parametre Space}
\chapterprecis{Identifying possible TLS candidates by comparing model results to experimental coupling variables such as the ground to first excited state energy splitting and dipole moment.}

\begin{figure}[htp]
\centering
\resizebox{0.8\columnwidth}{!}{\includestandalone{figures/mexhatproj}}
\caption[Potential Projections]{\label{fig:mexhatproj}\lin{1D} double wells (blue, solid) and harmonic wells (red, dashed) can be used to represent simple projections of a \lin{2D} potential onto the $x$ and $y$ axes. Left: two projected double wells is an example of a tetra-well. Right: a combination of one double well and a harmonic well reflects the hemi-tetra- case.}
\end{figure}

Possible sections:

\section{2D model}\label{sec:2d}
\subsection{Defects as Perturbed Bond Angles in a Lattice (1D)}
\subsection{TLS Defect Confined in 2 Dimensions}
\subsection{TLS Defect Confined in 3 Dimensions}
\subsection{Charge Dipoles}
\subsection{Qubit Coupling}
\section{3D}
\subsection{Methods of implementation}
\subsection{Potential Landscape analysis}
\subsubsection{Cluster analysis on a known grid}
\subsection{Comparative results with 2D model}