%\versoquote{When a person is determined to believe something, the very absurdity of the doctrine confirms them in their faith.}{Junius}
\chapter{Microscopic TLS Model}\label{ch:tls}
\chapterprecis{The Microscopic TLS Model chapter}%TODO: Fill this in.
\loftchap{Microscopic TLS Model}

\section{Concept and background}
\section{Potential configuration}

\begin{equation}
    H = -\frac{\hbar^2}{2m_{oxy}}\nabla^2+V(\mathbf{r})
    \label{eq:OHam}
\end{equation}

\begin{table}[h]
\caption[Streitz-Mintmire Pair Constants]{\label{tab:smconsts} Empirical constants used in the calculation of the Buckingham and Rydberg pair potentials in the Streitz-Mintmire formalism~\cite{Streitz1994,Gale2003}.}
\centering
\begin{tabular}{ c*{5}{r@{.}l} } \toprule
Pair & \multicolumn{2}{c}{A} & \multicolumn{2}{c}{$\rho$} & \multicolumn{2}{c}{B} & \multicolumn{2}{c}{C} & \multicolumn{2}{c}{$r_0$}  \\ \midrule
Al--Al & 4&474755 & 0&991317 & 0&159472 & 5&949144 & 3&365875 \\
Al--O & 62&933909 & 0&443658 & 0&094594 & 9&985407 & 2&358570 \\
O--O & 3322&784218 & 0&291065 & 1&865072 & 16&822405 & 2&005092 \\ \bottomrule
\end{tabular}
\end{table}

\begin{figure}[htp]
\centering
\resizebox{0.8\columnwidth}{!}{\includestandalone{figures/mexhatproj}}
\caption[Potential Projections]{\label{fig:mexhatproj}\lin{1D} double wells (blue, solid) and harmonic wells (red, dashed) can be used to represent simple projections of a \lin{2D} potential onto the $x$ and $y$ axes. Left: two projected double wells is an example of a tetra-well. Right: a combination of one double well and a harmonic well reflects the hemi-tetra- case.}
\end{figure}

\section{Solving Derivatives Numerically}\label{sec:numder}
The Streitz Mintmire potential \cite{Streitz1994} is an assemblage of many functional forms and integrals, which in total are not completely analytic over all regions of importance.
Therefore $\nabla^2$ in \cref{eq:OHam} will also require numerical treatment over a discrete grid of spatial coordinates.
A number of methods exist for this problem; most of which introduce errors from various approximations and platform limitations.
Models exist that effectively remove these errors, but are usually incredibly mathematically complex and problem specific: for example, a variational model which calculates an optimised three-finite-burn lunar escape trajectory~\cite{Ocampo2012}.

On the simpler end of the spectrum, finite difference algorithms are useful for boundary value problems (where forward and backward methods are usually applied), and for ordinary and partial differential equations.
If a first order ODE (or PDE) of the form $f(x)$ can be evaluated both left and right of $x$, the central difference method can be used where absciss\ae\ are chosen symmetrically about $x$, which takes the form
\nomdef{AODE}{ODE}{Ordinary Differential Equation}
\nomdef{APDE}{PDE}{Partial Differential Equation}
\begin{equation}
f\,'(x) \approx \frac{f(x+h)-f(x+h)}{2h},\label{eq:simplecdiff}
\end{equation}

where $h$ is some step size.
The step size controls the accuracy of the computed derivative, which is unfortunately bound by two factors.
If the step size is too small, numerical roundoff errors cause accuracy issues; on the other hand, step sizes which are too large sees mathematical truncation errors dominate.
Identifying an optimal balance for the step size is also dependent on the specific value of $f\,'(x)$ being calculated, which may not be appropriate for a similar set of values.
This problem, known as `the step size dilemma' has generated a number of investigations in an attempt to find a middle ground between simple finite difference methods with strong bounding conditions and the highly domain specific models like the variation model described above.

The complex step method (exploiting complex perturbations of the general Taylor series) yields no subtractive cancellation errors~\cite{Squire1998}, which exist in the real spaced Taylor series finite difference methods; \cref{eq:simplecdiff} is a simple example of this.
However, for many years this statement was only true for the first order derivative -- higher orders were shown to have as many cancellation errors as their finite difference counterparts.
This method was therefore excluded as a candidate for the model constructed for this thesis, as it would perform on par with standard finite difference methods for the second order derivative we wish to compute.
After much of the base code for this model had been constructed, a generalised, arbitrary order derivative complex step algorithm was published~\cite{Lantoine2012}.
This algorithm may improve the error contribution of the models' final implementation, although this has not been investigated.

A second possible successor to finite difference is automatic differentiation (AD).
\nomdef{AAD}{AD}{Automatic Differentiation}
The premise of this algorithm isn't necessarily mathematical in nature, but capitalises on the fact that computers are methodological reductionists and ultimately only execute simple arithmetic operations no matter how complicated the actual computation is.
Repeatedly applying the chain rule to these operations, AD can compute the derivative to working precision~\cite{Kedem1980}.
There a two methods to implement this algorithm; neither is straightforward.
One uses special AD preprocessors that analyse each function call, break it up instruction by instruction and generates a new function that computes derivatives.
Method two involves operator overloading that can generate code at compile time, which in a JIT accelerated environment like \texttt{Matlab} is unfeasible.

Both AD and complex step require access to functions at the source code level, meaning calls to third party libraries like \texttt{BLAS} and \texttt{LAPACK} (which are used in this model's implementation \xxx{section on lapack}) become increasingly over complicated.

For these reasons, it was decided to implement our model using a finite central difference method, paying close attention to the inherent error in exchange for relative computational simplicity.

\section[Understanding Central Differences]{Understanding the Central Difference Method\footnote{While most of this and the following section is standard cannon information for finite differences; it closely follows sections of \citeauthor{Mathur2012}~\cite{Mathur2012}, which will be formally introduced in \cref{sec:hopt}.}}\label{sec:ucdiff}

All finite difference methods involve truncating a Taylor series expansion of a function $f(x)$ about $x^*$ after a certain number of terms
\begin{equation}
    f(x) = f(x^*) + f^{\;\prime}(x)(x-x*) + \cdots + R_n.
\end{equation}

The discarded, higher order remainder terms $R_n$, are considered to contribute a negligible error to the approximation assuming a sufficiently small step size $h$.
\begin{equation}
    R_n = \sum_{m=n}^\infty \frac{f^{\;(m)}(x^*)}{m!}(x-x^*)^m
\end{equation}

It can be shown using the integral calculus derivation of the Taylor series that the $n^{th}$ derivative can be bound over the interval $[x^*, x]$, \ie $a\leq f^{\;(n)}(x) \leq~c$ \cite{Greenberg1978}.
Furthermore, $\exists b \in [a,c]$ such that the remainder can be written as
\begin{equation}
   R_n = b \frac{(x-x^*)^n}{n!}.
\end{equation}

The Intermediate Value Theorem can be invoked at this stage to posit some point $\xi \in [x^*,x]$ exists for which $f^{\;(n)}(\xi)$ will equal the unknown parameter $b$.
$R_n$ in this form is called the Lagrange remainder.
While there is no known method to determine a value of $\xi$ exactly for a general function, it is useful to express the $n^{th}$ order Taylor series in terms of the Lagrange remainder:
\begin{equation}
    f(x) = \sum_{m=0}^n \frac{f^{\;(m)}(x^*)}{m!}(x-x^*)^m + \frac{f^{\;(n+1)}(\xi)}{(n+1)!}(x-x^*)^{n+1}, \quad \xi \in [x^*,x].\label{eq:taylorrem}
\end{equation}

The most commonly used central difference formula is the first derivative of second order, which can be obtained by applying \cref{eq:taylorrem} at two absciss\ae\ of length $h$ from a sampling point in $f(x)$ and solving the simultaneous equation for $f^{\;\prime}(x)$.
\begin{align}
f(x+h) &= f(x) + f^{\;\prime}(x)h + \frac{f^{\;\prime\prime}(x)}{2}h^2 + \frac{f^{\;(3)}(\xi^+)}{2}h^3, \quad \xi^+ \in [x,x+h] \\
f(x-h) &= f(x) - f^{\;\prime}(x)h + \frac{f^{\;\prime\prime}(x)}{2}h^2 - \frac{f^{\;(3)}(\xi^-)}{2}h^3, \quad \xi^- \in [x-h,x] \\
f^{\;\prime}(x) &= \frac{f(x+h)-f(x-h)}{2h} - \frac{f^{\;(3)}(\xi^+)+f^{\;(3)}(\xi^-)}{2}\frac{h^2}{3!}\label{eq:cd12pm}
\end{align}

The error term in \cref{eq:cd12pm} contains the average of the third derivative evaluated the two unknown points $\xi^+$ and $\xi^-$, which are bounded inside the range $[x-h,x+h]$.
Applying the Mean Value Theorem and assuming that $f^{\;(3)}(x)$ is smooth over the bounded range, a value $\xi$ must exist between $\xi^+$ and $\xi^-$ which satisfies the average.
\begin{align}
f^{\;\prime}(x) &= \frac{f(x+h)-f(x-h)}{2h} - \frac{f^{\;(3)}(\xi)}{3!}h^2, \quad \xi \in [x-h,x+h]\label{eq:cd12t} \\
\mathcal{F}_2^{\,(1)}(x,h) &= \frac{f(x+h)-f(x-h)}{2h} + \mathcal{O}(h^2)\label{eq:cd12f}
\end{align}

Both equations above represent the first derivative of $f(x)$; although they differ in the sense that \cref{eq:cd12t} is still the true derivative and \cref{eq:cd12f} truncates the error term and represents it as an order of magnitude -- which is the essence of the finite difference approximation to the derivative.
The notation $\mathcal{F}_n^{\,(d)}(x,h)$ represents the general form of the approximation of the $d^{th}$ derivative of $f(x)$ of order $n$ using step size $h$, which can formerly be expressed as:
\nomdref{CFD}{$\mathcal{F}_n^{\,(d)}(x,h)$}{Finite difference approximation of the $d^{th}$ derivative of $f(x)$ of order $p = n+d-1$ using step size $h$}{ch:tls}
\begin{equation}
\mathcal{F}_n^{\,(d)}(x,h) = \frac{\Delta f_n^{\;(d)}(x,h)}{h^d} + \mathcal{O}(h^n).\label{eq:cdgeneral}
\end{equation}

The $\Delta f_n^{\;(d)}$ term describes the appropriate finite difference expression obtained from the set of \cref{eq:taylorrem} for particular values of $n$ and $d$.
Frequently used constructions of this form can be found in tables in books such as \citeauthor{Mathews2004} and web resources like \citeauthor{Holoborodko2009} without the need to derive them from first principles, although comprehensive error discussion is uncommon or oversimplified at best~\cite{Mathews2004,Holoborodko2009}.

Solving the Schr\"{o}dinger equation using the hamiltonian \cref{eq:OHam} requires a second derivative finite difference expression.
Truncating the Taylor series at $\mathcal{O}(h^2)$ is the simplest arrangement, which takes the form
\begin{equation}
f^{\;\prime\prime}(x) = \frac{f(x-h)-2f(x)+f(x+h)}{h^2} - \frac{f^{\;(4)}(\xi)}{12}h^2, \quad \xi \in [x-h,x+h]\label{eq:cd22t} \\
\mathcal{F}_2^{\,(2)}(x,h) = \frac{f(x-h)-2f(x)+f(x+h)}{h^2} + \mathcal{O}(h^2).\label{eq:cd22f}
\end{equation}

Before applying this method, a step size must be chosen; which relies on an accurate description of the competing error contributors.

\section{Contributors to the Step Size Dilemma}

\subsection{Truncation Error}

The truncation error in \cref{eq:cd22f} as $h\!\to\!0$ is not surprisingly $\mathcal{O}(h^2)\!\to\!0$, implying that $h$ should be as small as possible to maximise the accuracy of the calculation.
Limits can also be put on the true truncation error as well.
Even though $\xi$ is an unknown quantity, as the step size decreases, so does range in which $\xi$ exists.
Hence the limit of \cref{eq:cd22f} becomes
\begin{equation}
\lim_{h \to 0}\mathcal{O}(h^2) =  - \frac{f^{\;(4)}(x)}{12}h^2.
\end{equation}

This error differs for each formula, depending on the particular values of $n$ and $d$ (for example, the error in \labelcref{eq:cd12t} tends to $\cramped{-\frac{f^{\;(3)}(x)}{3!}h^2}$).
Using \cref{eq:cdgeneral} and \cref{eq:cd22f}, a general relationship between the true $d^{th}$ derivative and its finite difference approximation
\begin{equation}
 f_n^{\;(d)}(x) = \mathcal{F}_n^{\,(d)}(x,h) + C(x,h)h^n,\label{eq:cdadj}
\end{equation}

can be expressed with an undetermined truncation coefficient term $C(x,h)$, which is independent of the derivative parameter $d$.
This coefficient represented in Lagrange form is
\begin{equation}
C(x,h) = a_1\,f_n^{\;(n+d)}(\xi), \quad \xi \in [x-a_2h,x+a_3h],\label{eq:cdcxh}
\end{equation}

with the set of unknown constants $a$ also being dependent on the finite difference formula in question.
As the truncation coefficient's is dependent on $\xi$, its dependence on $h$ is removed as $h\!\to\!0$ (because $\xi\!\to\!x$).
Hence, for small values of $h$, a simplified coefficient can be defined as
\begin{equation}
 C_n(x) \equiv a_1\,f_n^{\;(n+d)}(x).
\end{equation}

Simplifying \cref{eq:cdadj} with this new coefficient, it is clear that this relationship takes the form of a Richardson extrapolation, therefore we can evaluate at two different step sizes $h_1$ and $h_2$ (where $h_1 > h_2$)
\begin{align}
f_n^{\;(d)}(x) &= \mathcal{F}_n^{\,(d)}(x,h_1) + C_n(x)h_1^n \\
f_n^{\;(d)}(x) &= \mathcal{F}_n^{\,(d)}(x,h_2) + C_n(x)h_2^n
\end{align}

and solving for $C_n(x)$ we find
\begin{equation}
 C_n = \frac{\mathcal{F}_n^{\,(d)}(x,h_2) - \mathcal{F}_n^{\,(d)}(x,h_1)}{h_1^n - h_2^n}.\label{eq:cdtrcoeff}
\end{equation}

This expression still assumes a small step size such that the approximated truncation coefficient stays independent of $h$ (and therefore constant for a valid range of steps).
The complete estimate for the truncation error of order $n$ can now be found via \cref{eq:cdadj}
\begin{equation}
\mathcal{TE}_n(x,h_1) = C_n h_1^n = \frac{\mathcal{F}_n^{\,(d)}(x,h_2) - \mathcal{F}_n^{\,(d)}(x,h_1)}{h_1^n - h_2^n}h_1^n.\label{eq:cdte}
\end{equation}

As much of this derivation requires a `sufficiently small' step size, it's imperative to consider the behaviour of the truncation error at larger values of $h$.
If $h$ increases, the approximate truncation coefficient can no longer be used and
the magnitude of the true coefficient $C(x,h)$ may become very large and unwieldily as there is no restriction of the values over $f^{\;(n+d)}(x\pm h_{large})$.
Additionally, $h^n$ in \cref{eq:cdte} increases, ultimately suggesting that a step size `too large' will also result in unpredictable truncation error values.

\subsection{Round Off Errors}\label{subsec:roundoff}

As mentioned in \cref{sec:numder}, numbers in a computer are represented with a fixed number of binary digits, and operations applied to them have an inherent loss of accuracy.
This phenomenon is designated the term round off error, as the extra digits that cannot be held in memory must be discarded and hence rounded to the nearest tolerance.

Round off error has been a hot topic of research since the invention of computers.
Notable checks on the accuracy of finite element methods were completed before the moon landings~\cite{Cyrus1968} and investigations after a Patriot missile defense system allowed a Scud missile to hit a barracks, killing 28 people; found round off via the differencing of floating point numbers introduced errors into the timing register when converting representations~\cite{Skeel1992}.
These relative errors caused by subtraction (called cancellation error) tend to decrease as $h$ is increased, thus to minimise this uncertainty $h$ should be as large as possible -- contrary to the truncation requirement of a small step size.
An upper bound on this error is given by
\begin{equation}
 \abs{(\alpha-\beta)_{true} - (\alpha-\beta)} \leq \delta \max(\abs{\alpha},\abs{\beta}),
\end{equation}

where $\alpha$ and $\beta$ are two fixed precision numbers, and $\delta$ indicates the precision of the calculation.
Usually, modern computer languages operate using standard double precision floating point numbers, thus $\delta = 2^{-53}$.

A second round off error type known as condition error creeps in when functions don't use machine precision floating point numbers in their internal routines.
For example, a third party function may approximate π to 3.14159, which would artificially round the result of whatever operation was being applied to a precision of $10^{-5}$.
As with cancellation error, an upper bound can also be formulated for condition error:
\begin{equation}
\abs{f(x)_{true} - f(x)}\leq \epsilon \abs{f(x)}.
\end{equation}

Here, $\epsilon$ is the magnitude of the most significant digit affected by the condition error.
For elementary operations $\epsilon$ should be equal to machine precision, although this cannot be assumed for all functions hence it is a value that should be calculated in general.

Finally, an error which is also important in this instance is named representational error.
Not all numbers can be accurately be represented in binary using a fixed number of digits.
Many people choose their step sizes in base 10 (\eg $1\ten{-2}$, $5\ten{-6}$ \etc), and what's most troubling about this is the fact that no negative power of 10 has an exact binary representation.
This error is small -- smaller that machine precision in fact, but it has non-trival and cumulative side effects.
If one avoids step sizes other than $2^\mathds{N}$ this error can be avoided completely.

With this information in hand, an upper bound to the total round off error can now be calculated:
\begin{equation}
 \abs{\mathcal{F}_n^{\,(d)}(x,h)_{true}-\mathcal{F}_n^{\,(d)}(x,h)} \leq \frac{\epsilon\abs{\mathcal{F}_\epsilon}+\delta\abs{\mathcal{F}_\delta}}{h^d}.
\end{equation}

The functions $\mathcal{F}_\epsilon$ and $\mathcal{F}_\delta$ are derived from $\Delta f_n^{\;(d)}$ (see \labelcref{eq:cdgeneral}).
Forms of \cref{eq:cd22f} for instance are
\begin{align}
\mathcal{F}_\epsilon &= \abs{f(x+h)} + \abs{2f(x)} + \abs{f(x-h)} \\
\mathcal{F}_\delta &= \max(\abs{f(x+h) + f(x-h)},\abs{2f(x)})
\end{align}

\subsection{Estimation of Total Error}

Using the equations for truncation and round off errors mentioned above, and assuming a small $h$ such that \cref{eq:cdtrcoeff} holds, the total error can be bounded by the expression
\begin{equation}
\abs{f^{\;(d)}(x)_{true}-\mathcal{F}_n^{\,(d)}(x,h)} \leq \frac{\epsilon\abs{\mathcal{F}_\epsilon}+\delta\abs{\mathcal{F}_\delta}}{h^d} + \abs{C_n}h^n
\end{equation}

Assuming a worse case scenario, the total error can therefore be described as
\begin{equation}
\mathcal{E}_n^{\,(d)}(x,h) = \frac{\epsilon\abs{\mathcal{F}_\epsilon}+\delta\abs{\mathcal{F}_\delta}}{h^d} + \abs{C_n}h^n\label{eq:cnetot}
\end{equation}

noting the dependence on $x$ comes from the functions $\mathcal{F}_\epsilon$ and $\mathcal{F}_\delta$, and that $\epsilon$ is still an unknown value.

\section{Calculating an Optimal Step Size}\label{sec:hopt}

With all of these caveats in mind, finding an optimal step size $h_{opt}$ is a troublesome undertaking; which is why the complex step and AD methods discussed in \cref{sec:numder} were developed.

A simple method proposed in \citeauthor{Gill1982} suggests minimising an expression similar to \cref{eq:cnetot} which trades off the truncation and round off errors ~\cite{Gill1982,Mathews2004}.
However, this method assumes the condition error $\epsilon$ is known and omits any cancellation error contributions.

Violation of monotonicity is an algorithm which attempts to ignore actually calculating any estimates for truncation and round off errors~\cite{Stepleman1979}. This method breaks down if \cref{eq:cdtrcoeff} doesn't hold and therefore is not a general solution.

Algorithms specifically designed for forward differences have also been discussed in the literature~\cite{Barton1992}, but are of no benefit for purposes herein.

The method which has been chosen to be implemented in this work is the algorithm from the PhD thesis of \textbf{Ravishankar Mathur}, which allows one to calculate an optimal step without \emph{a priori} knowledge of the condition error $\epsilon$.
Additionally, corrections to the the step size are introduce to account for the approximate nature of the truncation coefficient $C_n$, as well as estimations of step size validity and the maximal appropriate step size for a problem are examined~\cite{Mathur2012}.

As with \citeauthor{Gill1982}, \citeauthor{Mathur2012} starts by minimising the total error expression
\begin{equation}
\frac{\partial\mathcal{E}}{\partial h} = -d\frac{\epsilon\abs{\mathcal{F}_\epsilon}+\delta\abs{\mathcal{F}_\delta}}{h^{d+1}} + n\abs{C_n}h^{n-1} = 0
\end{equation}

and solving for $h$ to find
\begin{equation}
h_{opt,\mathcal{TE}} = \left[\frac{d}{n}\frac{1}{\abs{C_n}}\left(\epsilon\abs{\mathcal{F}_\epsilon}+
\delta\abs{\mathcal{F}_\delta}\right)\right]^{1/(n+d)}.\label{eq:cdhopt}
\end{equation}

Note that obtaining correct values of $\mathcal{F}_\epsilon$ and $\mathcal{F}_\delta$ require the optimal step size $h_{opt,\mathcal{TE}}$, thus an iterative method is required.
Once this value is known, \cref{eq:cdhopt} can be rearranged to find the condition error $\epsilon$.

The steps of the algorithm which computes these values (as well as discussions of further optimisations) can be found in Chapter 3 of \onlinecite{Mathur2012}.

\subsection{A Harmonic Approximation to Streitz Mintmire}\label{subsec:harmsm}

A complication arises when attempting to apply this algorithm to $\nabla^2$ in \cref{eq:OHam}, which more specifically should be written as $\cramped{\frac{\mathrm{d}^2\Psi}{\mathrm{d}x^2}}$; where $\Psi(x)$ is an eigenstate of $H$ and $E$ of the time-independent Schr\"odinger equation $H\ket{\Psi(x)} = E\ket{\Psi(x)}$. 
This eigenvalue equation is solved via direct diagonalisation of the hamiltonian matrix - requiring a descretised treatment of $\nabla^2$ before $\Psi(x)$ can be obtained.
The step size algorithm requires an analytical form of the function to which it is applied.
Unfortunately there are very few systems quantum mechanical systems with analytical solutions, and from that small pool, most are overly simplified constructions that could not exist physically.
The functional form of the wavefunction for a particle under the influence of the Streitz Mintmire potential is not one of these systems.
However, certain configurations of the potential approximate to a parabola-like shape (see section \xxx{Add in the mex hat section}x, \cref{fig:mexhatproj} and \cref{fig:smvh}) -- which has a simple analytic solution in the frame of\xxx{just for you Jared} the quantum harmonic oscillator that can serve as an analogue in this limit.

The hamiltonian of particle in a one dimensional quantum harmonic oscillator can be described as
\begin{equation}
\widehat{H} = \frac{\widehat{p}^2}{2m}+\frac{1}{2}m\omega^2\widehat{x}^2\label{eq:hamho}
\end{equation}

where $m$ is the particle's mass and $ω$ is the angular frequency of the oscillator.
Two operators, $\widehat{x} = x$ for position and $\cramped{\widehat{p} = -i\hbar \frac{\partial}{\partial x}}$ for momentum describe the complete Schr\"{o}dinger equation.
After separation of variables, the time independent form becomes
\begin{equation}
-\frac{\hbar^2}{2m}\frac{\partial^2 \Psi}{\partial x^2}+\frac{1}{2}m\omega^2x^2 \Psi = E\Psi,
\label{eq:hamti}
\end{equation}

with the family of solutions for the wavefunctions
\begin{align}\psi_n(x) &= \frac{1}{\sqrt{2^n\,n!}}\left(\frac{m\omega}{\hbar\pi}\right)^{1/4}e^{
- \frac{m\omega x^2}{2 \hbar}} H_n\left(\sqrt{\frac{m\omega}{\hbar}} x \right), &n = 0,1,2,\ldots \\
H_n(x) &= (-1)^n e^{x^2}\frac{\mathrm{d}^n}{\mathrm{d}x^n}\left(e^{-x^2}\right) \\
\therefore \psi_0(x) &= \left(\frac{m\omega}{\hbar\pi}\right)^{1/4}e^{
- \frac{m\omega x^2}{2 \hbar}} \label{eq:gshwfn}
\end{align}

where $\psi_0(x)$ \cref{eq:gshwfn} is the ground state wavefunction. Particle mass $m$ in our case is the mass of an oxygen atom, and the angular frequency $\omega$ is currently unknown.

\begin{figure}[htp]
\centering
\resizebox{\widefigure}{!}{\includestandalone{figures/smvh}}
\parbox{\widefigure}{\caption[Harmonic Approximation to Strietz Mintmire]{\label{fig:smvh}Calculated Streitz Mintmire \tikzlabel{0.25}{cylinder, minimum height=2cm,minimum width=0.3cm,fill=Set1-5-1} and Harmonic approximations \tikzlabel{0.25}{cylinder, minimum height=2cm,minimum width=0.3cm,fill=Set1-5-2} to the ground state wave function $\psi_0(x)$ (left) and potential $V(x)$ (right). The harmonic response is scaled via $\omega = 2.241\ten{13}$ rad/s using \cref{eq:gshwfn}.}}
\end{figure}

%below here is in need of adjustment.
which can now be used in conjunction with \cref{eq:cd22f} to compute $\nabla^2$ in \cref{eq:OHam}; evaluating at $x = 0$ for simplicity.
As discussed in \xxx{Introduction section describing magnitudes of parameters}, $V_{SM}$ should be measured in [μeV] with $x$ in [Å] to correspond to experimental results.

...
\Cref{fig:hopt3pt} displays the absolute value of the truncation error \cref{eq:cdte} over a range of possible step sizes \tikzlabel{0.4}{circle,fill=Set1-5-2}.
The smaller markers \tikzlabel{0.25}{circle,fill=Set1-5-2!50} represent steps which are not of the form $2^\mathds{N}$, which are shown for comparison but not used in the optimisation calculation.\xxx{I think this may actually be incorrect in some sense. Pages 47--49 talk about optimising $t^*$, which I've set to $1/(2^1) = 0.5$ and ignored all other possibilities, but the graph there suggests $t^*=0.75$ish. Perhaps one finds the optimal step with all values then chooses $h_{opt}$ as the closest power of 2?}
Round off errors dominate for small step sizes ($h \lesssim 10^{-4}$) with some step sizes resulting in completely invalid results (\ie $\mathcal{TE}_2 = 0$).
These steps are labelled \tikzlabel{0.25}{regular polygon, regular polygon sides=3,fill=Set1-5-2} and are scaled to $2\ten{-9}$ to be displayed on the graph.
As step size increases, truncation error dominates until the region $h \gtrsim 2$, where this error becomes invalid.
\begin{figure}[htp]
\centering
\resizebox{\columnwidth}{!}{\includestandalone{figures/hopt3pt}}
\caption[Step size optimisation of $f_2^{\;(2)}(x)$]{\label{fig:hopt3pt}Step size optimisation of $f_2^{\;(2)}(x)$ for step sizes $2^{-29} \leq h \leq 2^7$ \tikzlabel{0.4}{circle,fill=Set1-5-2}. Smaller markers \tikzlabel{0.25}{circle,fill=Set1-5-2!50} represent values of $\cramped{h \neq \left\{2^\mathds{N}\right\}}$. Steps which generate an invalid round off error (\ie $\mathcal{TE}_2 = 0$) are translated to $2\ten{-9}$ for display purposes and labelled \tikzlabel{0.25}{regular polygon, regular polygon sides=3,fill=Set1-5-2}. Two optimal step sizes are identified: $h_{opt,\mathcal{TE}}$ \tikzlabel{0.5}{circle,fill=Set1-5-1}, found using the optimal step algorithm~\cite{Mathur2012}, and $h_{opt,true}$ \tikzlabel{0.5}{circle,fill=Set1-5-3}, corrected by \cref{eq:hoptt}.}
\end{figure}

The step size optimisation algorithm~\cite{Mathur2012} initially chooses an uncorrected optimal step size \tikzlabel{0.5}{circle,fill=Set1-5-1} of $h_{opt,\mathcal{TE}} = 2^{-12} \simeq 2.441\ten{-4}$.
The truncation error \cref{eq:cdte} overestimates the true round off error contribution by a proportional amount given by $\cramped{t^* = (1+(1/t)^d)/(1-t^n)}$ such that the true optimal step is
\begin{equation}
h_{opt,true} = \left(\frac{1}{t^*}\right)^{1/(n+d)}h_{opt,\mathcal{TE}},\label{eq:hoptt}
\end{equation}

labelled as \tikzlabel{0.5}{circle,fill=Set1-5-3}.
The value of this corrected step is $h_{opt,true} \simeq 1.519\ten{-4}$, which does not have an accurate power of two representation.
As stated in \cref{subsec:roundoff}, representation error may be smaller than machine precision; although in particular instances this may have a non-trivial cumulative effect.

Using this result an the mass of an oxygen atom, a conversion factor for the optimal step sizes is yields values in relevant units
\begin{align}
\sqrt{\frac{\hbar}{m_{oxy}\omega}}10^{10} &= \sqrt{\frac{1.055\ten{-34}}{2.657\ten{-26}\cdot1.879\ten{14}}}10^{10} = 0.046\label{eq:confact} \\
h_{opt,\mathcal{TE}} &= 2^{-12}\cdot0.046 = 1.122\ten{-5} \simeq 2^{-17} & \left[\text{Å}\right]\\
h_{opt,true} &= 1.519\ten{-4}\cdot0.046 = 6.983\ten{-6} \simeq 2^{-17} & \left[\text{Å}\right]
\end{align}

In other words, on scales relevant to this thesis, the truncation error and true optimal step sizes are essentially equivalent.
Hence the optimal step size for the central difference second derivative of second order applied to \cref{eq:OHam} is
\begin{equation}
h_{opt} = 2^{-17} \simeq 7.629\ten{-6}.
\end{equation}

\section{Memory Concerns}\label{sec:memcons}

The complexity of numerical calculations are exacerbated by yet another constraint in the form of finite memory resources.
Ignoring many technical intricacies, a first order approximation to the required memory footprint just to store the numbers of one $9$ Å$^\text{\skolarlining 2}$ plane discretised via $h_{opt}$ requires $1,179,649^2$ double precision floats.
On a computer with 64 bit architecture, each double requires 8 bytes of memory, meaning our grid has a footprint of just over 11 Terabytes.

Computational resources of that magnitude are infeasible when a simple solution can both minimise memory requirements and increase the accuracy of the calculation simultaneously.
Recall the total error \cref{eq:cnetot}, which depends on $h$ via
\begin{equation}
\mathcal{E}_n^{\,(d)}(x,h) \propto \frac{1}{h^d} + h^n.
\end{equation}

For our purposes, $d$ is fixed at $2$, and $n$ was also chosen as $2$, but only due to the fact that this generates the simplest second derivative central difference formula.
Using order of magnitude arguments, a step size $h = 1\ten{-6}$ with an order parameter $n = 2$ contributes $\mathcal{O}(10^{-12})$ to the total error from the $h^n$ term.
If the order is increased to $n = 6$, a much larger step size $h = 1\ten{-2}$ yields the same $\mathcal{O}(10^{-12})$ contribution.

A back of the envelope calculation for a step size that large over the same range as above only requires $901^2$ doubles, which equates to a much more feasible memory requirement of 6.5 Megabytes.

\section{Second Derivative of Sixth Order}\label{sec:cd6o}

Using the formalism outline in \cref{sec:ucdiff}, an expression for the second derivative of sixth order can be calculated.
Starting with the Taylor series of the Lagrange remainder \cref{eq:taylorrem} using six absciss\ae\
\begin{align}
\MoveEqLeft f(x\pm h) = \nonumber\\
&f(x)\pm f\,'(x)h+\frac{f^{\;''}(x)h^2}{2!}\pm \frac{f^{\;(3)}(x)h^3}{3!}+\frac{f^{\;(4)}(x)h^4}{4!}\pm \frac{f^{\;(5)}(x)h^5}{5!}\nonumber\\
&+\frac{f^{\;(6)}(x)h^6}{6!}\pm \frac{f^{\;(7)}(x)h^7}{7!}+\frac{f^{\;(8)}(\xi^\pm)h^8}{8!}, \nonumber\\
&\qquad \qquad \xi^+ \in [x,x+h], \qquad \xi^- \in [x-h,x] \displaybreak[0]\\[0.2cm]
\MoveEqLeft f(x\pm 2h) = \nonumber\\
&f(x)\pm 2f\,'(x)h+\frac{4f^{\;''}(x)h^2}{2!}\pm \frac{8f^{\;(3)}(x)h^3}{3!}+\frac{16f^{\;(4)}(x)h^4}{4!}\pm \frac{32f^{\;(5)}(x)h^5}{5!}\nonumber\\
&+\frac{64f^{\;(6)}(x)h^6}{6!}\pm \frac{128f^{\;(7)}(x)h^7}{7!}+\frac{256f^{\;(8)}(\xi^\pm)h^8}{8!}, \nonumber\\
&\qquad \qquad \xi^+ \in [x,x+2h], \qquad \xi^- \in [x-2h,x] \displaybreak[0]\\[0.2cm]
\MoveEqLeft f(x\pm 3h) = \nonumber\\
&f(x)\pm 3f\,'(x)h+\frac{9f^{\;''}(x)h^2}{2!}\pm \frac{27f^{\;(3)}(x)h^3}{3!}+\frac{81f^{\;(4)}(x)h^4}{4!}\pm \frac{243f^{\;(5)}(x)h^5}{5!}\nonumber\\
&+\frac{729f^{\;(6)}(x)h^6}{6!}\pm \frac{2187f^{\;(7)}(x)h^7}{7!}+\frac{6561f^{\;(8)}(\xi^\pm)h^8}{8!}, \nonumber\\
&\qquad \qquad \xi^+ \in [x,x+3h], \qquad \xi^- \in [x-3h,x]
\end{align}

Followed by removing the odd degree terms
{\mathindent=0.5cm
\begin{align}
\MoveEqLeft f(x+h)+f(x-h) = \nonumber \\ &2f(x)+f^{\;''}(x)h^2+\frac{2f^{\;(4)}(x)h^4}{4!}+\frac{2f^{\;(6)}(x)h^6}{6!}+\frac{2f^{\;(8)}(\xi^\pm)h^8}{8!}\label{eq:fxh}\displaybreak[0]\\[0.2cm]
\MoveEqLeft f(x+2h)+f(x-2h) = \nonumber \\
&2f(x)+4f^{\;''}(x)h^2+\frac{32f^{\;(4)}(x)h^4}{4!}+\frac{128f^{\;(6)}(x)h^6}{6!}+\frac{512f^{\;(8)}(\xi^\pm)h^8}{8!}\label{eq:fx2h}\displaybreak[0]\\[0.2cm]
\MoveEqLeft f(x+3h)+f(x-3h) = \nonumber \\
&2f(x)+9f^{\;''}(x)h^2+\frac{162f^{\;(4)}(x)h^4}{4!}+\frac{1458f^{\;(6)}(x)h^6}{6!}+\frac{13122f^{\;(8)}(\xi^\pm)h^8}{8!}\label{eq:fx3h}
\end{align}
}

leaving a set of equations with fourth and sixth degree terms which need be eliminated in order to arrive at an equation similar to \cref{eq:cd22t}.
The equation $72\times$\cref{eq:fx2h}$-2\times$\cref{eq:fx3h}$-270\times$\cref{eq:fxh} eliminates these terms, arriving at
{\mathindent=0.3cm
\begin{equation}
f^{\;''}(x) = \frac{2f_{-3}-27f_{-2}+270f_{-1}-490f_{0}+270f_{1}-27f_{2}+2f_{3}}{180h^{2}}-\frac{f^{\;(8)}(\xi)h^6}{560}\label{eq:cd26t}\\
\mathcal{F}_2^{\,(6)}(x,h) = \frac{2f_{-3}-27f_{-2}+270f_{-1}-490f_{0}+270f_{1}-27f_{2}+2f_{3}}{180h^{2}} + \mathcal{O}(h^6),\label{eq:cd26f}
\end{equation}
}

where $\xi \in [x-3h,x+3h]$ and both the true and finite difference expressions are displayed in a simplified form.
The notation can be read as
\begin{equation}
f_k = f(x_k), \quad x_k=x^*+kh, \quad \left\{k \in \mathds{Z} \vert -(N-1)/2 \leq k \leq  (N-1)/2 \right\}
\end{equation}

for example $f_{-3} \equiv f(x-3h)$.
The form of \cref{eq:cd26f} adheres to \cref{eq:cdgeneral}, where
\begin{equation}
\Delta f_6^{\;(2)} = \frac{1}{180}2f_{-3}-27f_{-2}+270f_{-1}-490f_{0}+270f_{1}-27f_{2}+2f_{3}.\label{eq:f62x}
\end{equation}

Applying this equation is no different to \cref{eq:cd22f}, and the only additional overhead is the requirement of six absciss\ae\ rather than two.

An optimal step size for $\nabla^2$ in \cref{eq:OHam} can be found using  \cref{eq:cd26f}.
As with the calculation in \cref{subsec:harmsm} using \cref{eq:cd22f}, we will evaluate at $x = 0$ and apply the optimal step size algorithm~\cite{Mathur2012}; the results of which are shown in \cref{fig:hopt7pt}.
\begin{figure}[htp]
\centering
\resizebox{\columnwidth}{!}{\includestandalone{figures/hopt7pt}}
\caption[Step size optimisation of $f_2^{\;(6)}(x)$]{\label{fig:hopt7pt}Step size optimisation of $f_2^{\;(6)}(x)$ for step sizes $2^{-29}\!\leq\! h\! \leq\! 2^7$ \tikzlabel{0.4}{circle,fill=Set1-5-2}. Smaller markers \tikzlabel{0.25}{circle,fill=Set1-5-2!50} represent values of $\cramped{h \neq \left\{2^\mathds{N}\right\}}$. Two optimal step sizes are identified: $h_{opt,\mathcal{TE}}$ \tikzlabel{0.5}{circle,fill=Set1-5-1}, found using the optimal step algorithm~\cite{Mathur2012}, and $h_{opt,true}$ \tikzlabel{0.5}{circle,fill=Set1-5-3}, corrected by \cref{eq:hoptt}.}
\end{figure}

In comparison to the results of the second order method (see \cref{fig:hopt3pt}), which was calculated over the same step size range, it's clear that the estimations in \cref{sec:memcons} hold.
The absolute error $\abs{\mathcal{TE}_6}$ minimum is in fact three orders smaller, and the optimal step sizes two orders larger (in dimensionless units at least).
Applying the dimensionless unit conversion \cref{eq:confact}, the step sizes become
\begin{align}
h_{opt,\mathcal{TE}} &= 2^{-6}\cdot0.046 = 7.181\ten{-4} \simeq 2^{-11} & \left[\text{Å}\right]\\
h_{opt,true} &= 1.275\ten{-2}\cdot0.046 = 5.861\ten{-4} \simeq 2^{-11} & \left[\text{Å}\right]
\end{align}

again showing equivalency on the angstrom scale, therefore the optimal step size for the central difference second derivative of sixth order applied to \cref{eq:OHam} is
\begin{equation}
h_{opt} = 2^{-11} \simeq 4.883\ten{-4}.
\end{equation}

A step of this size for a $9$ Å$^\text{\skolarlining 2}$ grid requires $18,433^2$ doubles with a memory footprint of around 2.7 Gigabytes.

\section{Condition and Total Error Calculation}

Now that the optimal step size has been calculated, both the condition error $\epsilon$ and total error $\mathcal{E}_6^{\,(2)}(x,h)$ for the sixth order form can be found.

Recall the optimal step size formula \cref{eq:cdhopt}, which can now be rearranged in terms of the condition error
\begin{equation}
\epsilon = \frac{1}{\abs{\mathcal{F}_\epsilon}}\left(\frac{n}{d}\abs{C_n}h_{opt,\mathcal{TE}}^{n+d}-\delta\abs{\mathcal{F}_\delta}\right)
\end{equation}

and the remaining unknowns can also be calculated.
For ease of calculation, the dimensionless unit form of the step size will be used: $h_{opt,\mathcal{TE}} = 2^{-6}$.
$C_n$ \cref{eq:cdtrcoeff} is calculated during the optimal step algorithm~\cite{Mathur2012} when the steps $h_1$ and $h_2$ are found to be in the valid truncation error range.
The absolute value of which is $\abs{C_n} = 0.141$.
Functional forms of the condition error $\mathcal{F}_\epsilon$ and cancellation error $\mathcal{F}_\delta$ coefficients are generated from \cref{eq:f62x}
{\mathindent=0.4cm
\begin{align}
\mathcal{F}_\epsilon &= \frac{1}{180}2\abs{f_{-3}} + 27\abs{f_{-2}} + 270\abs{f_{-1}} + 490\abs{f_{0}} + 270\abs{f_{1}} + 27\abs{f_{2}} + 2\abs{f_{3}} \\
\mathcal{F}_\delta &= \frac{1}{180}\max(\abs{27f_{-2} + 490f_{0} + 27f_{2}},\abs{2f_{-3} + 270f_{-1} + 270f_{1} + 2f_{3}}),
\end{align}
}

yielding a value for the condition error of $\epsilon = 1.530\ten{-18}$.\xxx{This is a value for the harmonic oscillator wavefunction. Can we connect it to SM? Otherwise this entire section is probably pointless. Converting to real units isn't necessarily making sense atm either.}
This result is beneath machine precision ($\delta = 2^{-53} \simeq 1.110\ten{-16}$), which is a good result.

The total error \cref{eq:cnetot} is now a trivial calculation, using the above values $\mathcal{E}_6^{\,(2)}(0,h_{opt,\mathcal{TE}}) = 8.193\ten{-12}$.

\section{Maximum Step Size}

Whilst the 2.7 Gigabyte memory footprint calculated for a $9$ Å$^\text{\skolarlining 2}$ plane using the sixth order second derivative in \cref{sec:cd6o} is feasible only if one plane was required.
Below in \xxx{section on calculating the wavefunctions}, the need for calculating at least 6 excited state wavefunctions using direct matrix diagonalisation.
This, along with many other computational overheads sees the memory requirement balloon and the problem again becomes intractable.

It is shown in \onlinecite{Mathur2012} that the optimal step size is valid over the range $x \in [x^*-a_2h_{max},x^*+a_3h_{max}]$ if $f^{\;(n+d)}(x)$ can be shown to be constant over the same range.
This is in turn proven by showing $f^{\;(n+d)}(\xi)$ is constant over $[h_{opt},h_{max}]$.
The derivation is related to \cref{eq:cdcxh}, where $a_2$ and $a_3$ originate.
The value $h_{max}$ is simply the point at which the truncation error becomes valid as step size decreases.
Thus $h_{max} = 1$ (in dimensionless units) for the sixth order second derivative from \cref{fig:hopt7pt} so long as $f^{\;(8)}(\xi)$ is constant over $[2^{-6},2^{0}]$. \xxx{Fun fact: It varies over this entire region}


