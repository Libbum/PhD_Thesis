\chapter{Microscopic TLS model}
\section{Concept and background}
\section{Potential configuration}


\begin{equation}
    H = -\frac{\hbar^2}{2m_{oxy}}\nabla^2+V(\mathbf{r})
    \label{eq:OHam}
\end{equation}


\section{Finite Grid}
The Streitz Mintmire potential \cite{Streitz1994} is an assemblage of many functional forms and integrals, some of which can only be solved numerically. Therefore $\nabla^2$ in \cref{eq:OHam} will also require numerical treatment via a finite difference method. 

Finite difference algorithms are useful for boundary value problems (where forward and backward methods are usually applied), and for ordinary and partial differential equations. If the ODE (or PDE) of the form $f(x)$ can be evaluated both left and right of $x$, the central difference method can be used where abscissas are chosen symmetrically about $x$, which takes the form

\begin{equation}
f\,'(x) \approx \frac{f(x+h)-f(x+h)}{2h},
\end{equation}
where $h$ is some step size. For an overview of these algorithms, see \onlinecite{Mathews2004}.
  
Usually methods of low orders (\eg $\mathcal{O}(h^2)$) are sufficient to achieve a comparable results to the original ODE, although this was not the case for some responses of Streitz Mintmire. A higher order treatment was required.

\subsection{Derivation}
Following the discussion in \onlinecite{Holoborodko2009}: to calculate the numerical derivative of the function $f(x)$ one can approximate the local region about any point $x^*$ by an appropriate polynomial $P_m(x)$. Derivatives of the polynomial are assumed to be equal to the derivative of the original function: $f\,'(x^*)\approx P\,'_m(x^*)$ and $f\,''(x^*)\approx P\,''_m(x^*)$ \etc

Sampling $f(x)$ at $N$ abscissas (where $N$ is odd) equidistant from $x^*$ such that 

\begin{equation}
f_k = f(x_k), \quad x_k=x^*+kh, \quad \left\{k \in \mathds{Z} \vert -(N-1)/2 \leq k \leq  (N-1)/2 \right\}
\end{equation}

and interpolating points $\{(x_k,f_k)\}$ by a polynomial of degree $N-1$
\begin{equation}
P_{N-1}(x)=\sum_{j=0}^{N-1} a_j x^j,
\end{equation}

a solution can be found by obtaining the set of coefficients $\{a_j\}$ for the system of linear equations
\begin{equation}
\left\{P_{N-1}(x_k)=f_x\right\}.
\label{eq:ploylinset}
\end{equation}

For reasons outlined \emph{vide infra} in \cref{sec:cdifferr}, a centered formula of order $\mathcal{O}(h^6)$ is needed to obtain the error tolerance required to solve \cref{eq:OHam}, thus a $N=7$ point calculation. 

The interpolating parabola \cref{eq:ploylinset} are

\begin{align}
&\begin{cases}
    P_7(-3h) &= \quad f_{-3}\\
    P_7(-2h) &= \quad f_{-2}\\
    P_7(-h) &= \quad f_{-1}\\
    P_7(0) &= \quad f_{0}\\
    P_7(h) &= \quad f_{1}\\
    P_7(2h) &= \quad f_{2}\\
    P_7(3h) &= \quad f_{3}
  \end{cases} \nonumber \displaybreak[0]\\[0.2cm]
  \Rrightarrow
  &\begin{cases}
    a_0 - 3a_1h + 9a_2h^3 - 27a_3h^3 + 81a_4h^4 - 243a_5h^5 + 729a_6h^6 &= f_{-3}\\
    a_0 - 2a_1h + 4a_2h^2 - 8a_3h^3 + 16a_4h^4 - 32a_5h^5 + 64a_6h^6 &= f_{-2}\\
    a_0 - a_1h + a_2h^2 - a_3h^3 + a_4h^4 - a_5h^5 + a_6h^6 &= f_{-1}\\
    a_0 &= f_{0}\\
    a_0 + a_1h + a_2h^2 + a_3h^3 + a_4h^4 + a_5h^5 + a_6h^6 &= f_{1}\\
    a_0 + 2a_1h + 4a_2h^2 + 8a_3h^3 + 16a_4h^4 + 32a_5h^5 + 64a_6h^6 &= f_{2}\\
    a_0 + 3a_1h + 9a_2h^3 + 27a_3h^3 + 81a_4h^4 + 243a_5h^5 + 729a_6h^6 &= f_{3}
  \end{cases}
\end{align}

Solving this set of simultaneous equations for $\{a_j\}$ yields
\begin{equation}
\begin{cases}
a_0 &= f_{0}\\[0.3cm]
a_1 &= \dfrac{-f_{-3}+9f_{-2}-45f_{-1}+45f_{1}-9f_{2}+f_{3}}{12h}\\[0.3cm]
a_2 &= \dfrac{2f_{-3}-27f_{-2}+270f_{-1}-490f_{0}+270f_{1}-27f_{2}+2f_{3}}{360h^2}\\[0.3cm]
&\vdots
\end{cases}
\end{equation}

Assuming $x^* = 0$ for simplicity, it's clear that $f\,'(0) \approx P\,'_7(0) = a_1$ and therefore

\begin{equation}
\begin{split}
f\,''(0)&\approx P\,''_7(0)=2a_2\\
&=\frac{2f_{-3}-27f_{-2}+270f_{-1}-490f_{0}+270f_{1}-27f_{2}+2f_{3}}{180h^{2}}+\mathcal{O}(h^{6}).
\end{split}
\end{equation}

Higher order central difference is given by

\begin{equation}
\delta^n_h[f](x) = \sum_{i = 0}^{n} (-1)^i \binom{n}{i} f\left(x + \left(\frac{n}{2} - i\right) h\right)
\end{equation}

Later, for now - here it is: Centered formula of $\mathcal{O}(h^{6})$.
\begin{equation}
f\,''(x_0)=\frac{2f_{-3}-27f_{-2}+270f_{-1}-490f_{0}+270f_{1}-27f_{2}+2f_{3}}{180h^{2}}+\mathcal{O}(h^{6})
\end{equation}

where $f_k = f(x_0 + kh)$.

\subsection{Error Analysis}\label{sec:cdifferr}
An accurate error in computing value is unobtainable directly. However, in particular limits the potential behaves as a harmonic oscillator - which has a simple analytic solution and can serve as an analogue.

The hamiltonian of particle in a one dimensional quantum harmonic oscillator is

\begin{equation}
\widehat{H} = \frac{\widehat{p}^2}{2m}+\frac{1}{2}mw^2\widehat{x}^2
\end{equation}

where $m$ is the particle's mass and $ω$ is the angular frequency of the oscillator. Two operators, $\widehat{x} = x$ for position and $\widehat{p} = -i\hbar \frac{\partial}{\partial x}$ for momentum describe the complete Schr\"{o}dinger equation, which (after separation of variables) takes its time independent form:

\begin{equation}
-\frac{\hbar^2}{2m}\frac{\partial^2 \Psi}{\partial x^2}+\frac{1}{2}mw^2x^2 \Psi = E\Psi.
\end{equation}

...

We differentiate to the 8th order

\begin{equation}
\Psi_0^{(8)}(x) = \left(\frac{1}{\pi}\right)^{\frac{1}{4}}\left(x^8-28x^6+210x^4-420x^2+105\right)e^{-\frac{x^2}{2}}
\label{eq:7pt}
\end{equation}

To find $M$ we know that
\begin{equation}
 \left|\Psi_0^{(8)}(x)\right| \leq M
\end{equation}
so
\begin{equation}
M = \mathrm{max}\left(\left|\Psi_0^{(8)}(x)\right|, \quad -4.5 \leq x \leq 4.5\right) = 78.8682
\end{equation}
for our purposes.


Now, take \cref{eq:7pt} and do something (figure this out later)?

\begin{equation}
180f^{\prime\prime}(x)h^2+\frac{9}{28}f^{(8)}(x)h^8
\end{equation}
\begin{equation}
  f^{\prime\prime}(x)=-\frac{f^{(8)}(x)h^6}{560}
\end{equation}

Thus
\begin{equation}
  E(f,h) = \frac{1088\epsilon}{180h^2}-\frac{f^{(8)}(\epsilon)h^6}{560}
\end{equation}
where the left term is the round off error and the right term is the truncation error and the equation to minimise is
\begin{equation}
  \left|E(f,h)\right| \leq \frac{27^2\epsilon}{45h^2}-\frac{Mh^6}{560}
\end{equation}
therefore (if $g(x) = 0$)
\begin{equation}
  \frac{27^2\epsilon}{45h^2}=\frac{Mh^6}{560}
\end{equation}
simplifying to $152320\epsilon = 45Mh^8$ and solving for $h$ we find
\begin{equation}
  h = \left(\frac{30464\epsilon}{9M}\right)^{1/8} \approx 0.1100149
\end{equation}
if $\epsilon = 0.5\;\times\;10^{-9}$ and $M = 78.8682$.

So our optimal step size is $0.1$.

What's actually more useful for us is to know the precision of 7pt central difference with the current step size, so
\begin{equation}
\epsilon = \frac{\left(9*78.8682*h^8\right)}{30464} = 2.33\;\times\;10^{-10}
\end{equation}