%For now, this is essentially a sandbox for new design additions.
\versoquote{The task must be made difficult, for only the difficult inspires the noble-hearted.}{Søren Kierkegaard}
%\versoimage %Will generate a black page with a tikz image
\chapter{Junction Models}
%This is from memoir, it will write this text in both the ToC and at the top of the chapter.
\chapterprecis{Amorphous insulating barriers (either in the form of Josephson junctions or simply a native oxide) form an integral part of superconducting circuits, so it comes as no surprise that TLSs are often considered to be an important source of noise in these circuits.}

%\marginpar{This is a test margin paragraph}
\section{How a junction is built in experiment}
Bistable defects \sidepar{This is a side par, you can do more stuff with it, but it may overlap so be careful.} in glasses and amorphous solids in general have been understood for some time [17].
Amorphous insulating barriers (either in the form of Josephson junctions or simply a native oxide) form an integral part of superconducting circuits, so it comes as no surprise that TLSs are often considered to be an important source of noise in these circuits [1, 2, 9]. Developments in controllable
qubit architecture (charge, flux and phase) has\footnote{margin footnote with a few more words in it to make it a bit better.} enabled the study of so-called
‘strongly coupled defects’ [4, 6, 7]. These defects have comparable resonance fre-
quencies to the qubit circuit and coupling strengths as well as decoherence times
long enough to allow\footnote{a second} coherent oscillations between the qubit and TLS. Probing
individual defects has promoted their bistable\footnote{and third} nature from hypothesis to observ-
able fact, as well as providing clues to their microscopic origin.
\section{DFT and MM theory}
\section{Model construction}
\subsection{Al slabs in DFT}
\subsection{Amorphous \texorpdfstring{\ce{AlO_{x}}}{AlOₓ} in MM}
\subsection{DFT MD and optimisation}
\section{Analysis of structures}
\subsection{Stoichiometry}
\subsection{Density}
\subsection{\texorpdfstring{$G(r)$}{G(r)}}
In cited \cite{Xu1997} 