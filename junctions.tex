%\versoquote{The task must be made difficult, for only the difficult inspires the noble-hearted.}{Søren Kierkegaard}
\chapter{Junction Models}
\loftchap{Junction Models}
\chapterprecis{Junction models, how they're built and shit.}

\thought{Here is a section that} starts with a thought that is in small caps.
\sidepar{This is a side par, you can do more stuff with it, but it may overlap so be careful.}

\section{How a junction is built in experiment}

This section is currently lorem ipsum.
Bistable defects in glasses and amorphous solids in general have been understood for some time.
Amorphous insulating barriers (either in the form of Josephson junctions or simply a native oxide) form an integral part of superconducting circuits, so it comes as no surprise that TLSs are often considered to be an important source of noise in these circuits. 
Developments in controllable qubit architecture (charge, flux and phase) has\footnote{margin footnote with a few more words in it to make it a bit better.} enabled the study of so-called ‘strongly coupled defects’.
These defects have comparable resonance frequencies to the qubit circuit and coupling strengths as well as decoherence times long enough to allow\footnote{a second \the\marginparwidth, \the\marginparsep, \the\marginparpush} coherent oscillations between the qubit and TLS.
Probing individual defects has promoted their bistable\footnote{and third} nature from hypothesis to observable fact, as well as providing clues to their microscopic origin.

\section{DFT and MM theory}
\section{Model construction}
\subsection{Al slabs in DFT}
\subsection{Amorphous \texorpdfstring{\ce{AlO_{x}}}{AlOₓ} in MM}
\subsection{DFT MD and optimisation}
\section{Analysis of structures}
\subsection{Stoichiometry}
\subsection{Density}
\subsection{\texorpdfstring{$G(r)$}{G(r)}}
\begin{figure}[htp]
\centering
\resizebox{\textwidth}{!}{\includestandalone{figures/gr}}
\caption[Radial Distribution Function]{\label{fig:gofr}Radial distribution function.}
\end{figure}