%For now, this is essentially a sandbox for new design additions.
\versoquote{The task must be made difficult, for only the difficult inspires the noble-hearted.}{Søren Kierkegaard}
%\versoimage %Will generate a black page with a tikz image
\chapter{Junction Models}
%This is from memoir, it will write this text in both the ToC and at the top of the chapter.
\chapterprecis{Amorphous insulating barriers (either in the form of Josephson junctions or simply a native oxide) form an integral part of superconducting circuits, so it comes as no surprise that TLSs are often considered to be an important source of noise in these circuits.}

%\marginpar{This is a test margin paragraph}
\section{How a junction is built in experiment}
Bistable defects \sidepar{This is a side par, you can do more stuff with it, but it may overlap so be careful.} in glasses and amorphous solids in general have been understood for some time [17].
Amorphous insulating barriers (either in the form of Josephson junctions or simply a native oxide) form an integral part of superconducting circuits, so it comes as no surprise that TLSs are often considered to be an important source of noise in these circuits [1, 2, 9]. Developments in controllable
qubit architecture (charge, flux and phase) has\footnote{margin footnote with a few more words in it to make it a bit better.} enabled the study of so-called
‘strongly coupled defects’ [4, 6, 7]. These defects have comparable resonance fre-
quencies to the qubit circuit and coupling strengths as well as decoherence times
long enough to allow\footnote{a second \the\marginparwidth, \the\marginparsep, \the\marginparpush} coherent oscillations between the qubit and TLS. Probing
individual defects has promoted their bistable\footnote{and third} nature from hypothesis to observ-
able fact, as well as providing clues to their microscopic origin.
\section{DFT and MM theory}
\section{Model construction}
\subsection{Al slabs in DFT}
\subsection{Amorphous \texorpdfstring{\ce{AlO_{x}}}{AlOₓ} in MM}
\subsection{DFT MD and optimisation}
\section{Analysis of structures}
The Streitz Mintmire potential \cite{Streitz1994} is an assemblage of many functional forms and integrals, some of which can only be solved numerically.
Therefore $\nabla^2$ in \cref{eq:OHam} will also require numerical treatment via a finite difference method.

Finite difference algorithms are useful for boundary value problems (where forward and backward methods are usually applied), and for ordinary and partial differential equations.
If the ODE (or PDE) of the form $f(x)$ can be evaluated both left and right of $x$, the central difference method can be used where abscissas are chosen symmetrically about $x$, which takes the form

\begin{equation}
f\,'(x) \approx \frac{f(x+h)-f(x+h)}{2h},\label{eq:simplecdiff}
\end{equation}
where $h$ is some step size.
\begin{marginfigure}
\resizebox{\marginparwidth}{!}{\includestandalone{figures/mexhatproj}}
\end{marginfigure}
For an overview of these algorithms, see \onlinecite{Mathews2004}.

Usually methods of low orders (\eg $\mathcal{O}(h^2)$) are sufficient to achieve a comparable results to the original ODE, although this was not the case for some responses of Streitz Mintmire.
A higher order treatment was required.

\subsection{Stoichiometry}
\lin{1D} model vs 2D model
\subsection{Density}
\subsection{\texorpdfstring{$G(r)$}{G(r)}}
In cited \cite{Xu1997} 