\chapter*{Summary}
\addcontentsline{toc}{chapter}{Summary}
\vspace{-1.2cm}
\thought{Constructing circuits which invoke the} tunnelling effect of Josephson junctions in the superconducting regime manifest controllable, quantum properties.
The development of a quantum computer employing these phenomena is just one of the many advantages to be gained from building such devices.

However, the use of quantum bits for computation is dependent on the ability to operate them in an essentially isolated environment.
The phenomenon of decoherence refers to the instability of a quantum state of a system when it interacts with the surrounding environment.
Superconducting qubits are sensitive to decoherence mechanisms within the readout leads connecting to the device, and more importantly from the materials from which they are manufactured.
Removal or control of these imperfections is required before quantum computers using superconducting circuit architecture can be realised.

One identifiable noise source is the so called `strongly coupled' two-level system (TLS).
Comparable resonance frequencies to the qubit; strong coupling strengths and decoherence times long enough to allow coherent oscillations between the qubit and TLS have been experimentally measured.
The premise of this thesis is that positional anharmonicity of oxygen atoms arises within the \ce{AlO_x} barrier of the Josephson junction solely due to its amorphous construction.
This ansatz allows the existence of various spatial configurations throughout the layer, causing unique TLS properties based solely on atomic positions and rotation in relation to the external electric field.

To validate this conjecture, Josephson junction models are constructed using a hybrid \textit{ab initio} and molecular mechanics approach, with various stoichiometry and density properties to reflect experimental observations of the barrier.
The resultant atomic positions provide input conditions through a Voronoi classification scheme to a framework describing an oxygen atom that has the capacity to become spatially delocalised as bonds perturb away from a crystalline structure.
A direct diagonalisation method is developed for low dimensional descriptions and a Wick-rotated time-dependent Schrödinger equation implementation is used for three dimensional investigations.

Calculated properties are compared to many current experimental strongly coupled TLS measurements in phase qubits, which shows the models' capacity to explain how an oxygen atom can generate a large dipole and appropriate ground to first excited state splitting values expected of a TLS by merely migrating from its preferred lattice position.

